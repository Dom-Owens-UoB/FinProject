\documentclass[]{article}
\usepackage{lmodern}
\usepackage{amssymb,amsmath}
\usepackage{ifxetex,ifluatex}
\usepackage{fixltx2e} % provides \textsubscript
\ifnum 0\ifxetex 1\fi\ifluatex 1\fi=0 % if pdftex
  \usepackage[T1]{fontenc}
  \usepackage[utf8]{inputenc}
\else % if luatex or xelatex
  \ifxetex
    \usepackage{mathspec}
  \else
    \usepackage{fontspec}
  \fi
  \defaultfontfeatures{Ligatures=TeX,Scale=MatchLowercase}
\fi
% use upquote if available, for straight quotes in verbatim environments
\IfFileExists{upquote.sty}{\usepackage{upquote}}{}
% use microtype if available
\IfFileExists{microtype.sty}{%
\usepackage{microtype}
\UseMicrotypeSet[protrusion]{basicmath} % disable protrusion for tt fonts
}{}
\usepackage[margin=1in]{geometry}
\usepackage{hyperref}
\hypersetup{unicode=true,
            pdftitle={Finance Project: Forecasting the S\&P 500},
            pdfauthor={Dom Owens},
            pdfborder={0 0 0},
            breaklinks=true}
\urlstyle{same}  % don't use monospace font for urls
\usepackage{graphicx,grffile}
\makeatletter
\def\maxwidth{\ifdim\Gin@nat@width>\linewidth\linewidth\else\Gin@nat@width\fi}
\def\maxheight{\ifdim\Gin@nat@height>\textheight\textheight\else\Gin@nat@height\fi}
\makeatother
% Scale images if necessary, so that they will not overflow the page
% margins by default, and it is still possible to overwrite the defaults
% using explicit options in \includegraphics[width, height, ...]{}
\setkeys{Gin}{width=\maxwidth,height=\maxheight,keepaspectratio}
\IfFileExists{parskip.sty}{%
\usepackage{parskip}
}{% else
\setlength{\parindent}{0pt}
\setlength{\parskip}{6pt plus 2pt minus 1pt}
}
\setlength{\emergencystretch}{3em}  % prevent overfull lines
\providecommand{\tightlist}{%
  \setlength{\itemsep}{0pt}\setlength{\parskip}{0pt}}
\setcounter{secnumdepth}{0}
% Redefines (sub)paragraphs to behave more like sections
\ifx\paragraph\undefined\else
\let\oldparagraph\paragraph
\renewcommand{\paragraph}[1]{\oldparagraph{#1}\mbox{}}
\fi
\ifx\subparagraph\undefined\else
\let\oldsubparagraph\subparagraph
\renewcommand{\subparagraph}[1]{\oldsubparagraph{#1}\mbox{}}
\fi

%%% Use protect on footnotes to avoid problems with footnotes in titles
\let\rmarkdownfootnote\footnote%
\def\footnote{\protect\rmarkdownfootnote}

%%% Change title format to be more compact
\usepackage{titling}

% Create subtitle command for use in maketitle
\providecommand{\subtitle}[1]{
  \posttitle{
    \begin{center}\large#1\end{center}
    }
}

\setlength{\droptitle}{-2em}

  \title{Finance Project: Forecasting the S\&P 500}
    \pretitle{\vspace{\droptitle}\centering\huge}
  \posttitle{\par}
    \author{Dom Owens}
    \preauthor{\centering\large\emph}
  \postauthor{\par}
      \predate{\centering\large\emph}
  \postdate{\par}
    \date{18/11/2019}


\begin{document}
\maketitle

\hypertarget{factor-analysis}{%
\subsection{Factor Analysis}\label{factor-analysis}}

Within this project, we will use \textbf{Factor Analysis} as a dimension
reduction technique for our mutlivariate time series \(\mathbf{x}_t\).
The problem is treated as follows:

We wish to reduce the dimension of an observable random vector
\(\mathbf{x} \in \mathbb{R}^p\) to a smaller vector
\(\mathbf{x} \in \mathbb{R}^m\) of latent variabes, where \(m << p\).

We do this by expressing each \(x_i\) as a linear combination of the
factors:
\[x_i = \lambda_{i1} f_1 + ... + \lambda_{im} f_m + \epsilon_i \hspace{30pt} \mathbf{x} = \Lambda \mathbf{f} + \boldsymbol{\epsilon}\]

Here, \(\Lambda\) are the \textbf{factor loadings}, and
\(\boldsymbol{\epsilon}\) are errors.

We make the following assumptions:

\begin{itemize}
\item
  \(E(\boldsymbol{\epsilon}) = \boldsymbol{0}\),
  \(E(\boldsymbol{f}) = \boldsymbol{0}\),
  \(E(\boldsymbol{x}) = \boldsymbol{0}\) (WLOG)
\item
  \(E(\boldsymbol{\epsilon \epsilon^T}) = \boldsymbol{\Psi}\) is a
  diagonal matrix
\item
  \(E(\boldsymbol{f f^T}) = \boldsymbol{I}_m\), so that the factors are
  independent
\item
  For inferential purposes, we make distributional assumptions on
  \(\boldsymbol{f}\) or \(\boldsymbol{x}\) (often multivariate
  normality)
\end{itemize}

We will be working with \textbf{time series} data and models, meaning
our observations \(\mathbf{x}_t\) are indexed in time by
\(t \in \{ 0, 1, ... T \}\). We further assume that

\begin{itemize}
\item
  The \textbf{Covariance} matrix \(\Sigma\) is constant with respect to
  \(t\)
\item
  (further assumptions - second order stn.ry, differences etc.)
\end{itemize}

Indeed, underlying the whole idea of forecasting financial markets is
the \textbf{Big Assumption}, which is that economic activity in the near
future will closely resemble economic activity in the past.

\hypertarget{estimation}{%
\subsubsection{Estimation}\label{estimation}}

As opposed to the similar-looking regression problem
\[\boldsymbol{y} = X \boldsymbol{\beta} + \boldsymbol{\epsilon} \], in
which the desgin matrix \(X\) is known, we know neither
\(\boldsymbol{f}\) nor \(\Lambda\); hence, any ``best-fit'' solutions
will not be unique.

For conducting estimation in practice, we often find \(\hat{\Lambda}\)
and \(\hat{\Phi}\), then find \(\boldsymbol{\hat{f}}\). In disciplines
such as finance and economics, the factors can be pre-specified
according to theoretical justifications (see the {[}Fama-French factor
model{]}\{\url{https://www.investopedia.com/terms/f/famaandfrenchthreefactormodel.asp}\}
); we will use a mathematical approach instead.

One way of finding estimates leverages Principal Components Analysis
(PCA), called \textbf{Principal Factor Analysis (PFA)}. is what we will
use in this project.

We work with the covariance and sample covariances matrices
\[ \Sigma = \Lambda \Lambda^T + \Psi \text{ and } 
S = \frac{1}{n}XX^T\] Suppose that we have the principal components
decomposition \[ \mathbf{x}_t = A^T \mathbf{z}_t  \] where
\(A \in \mathbb{R}^{p \times n}\) is a matrix consisting of eigenvectors
\(\boldsymbol{\alpha_i}\), each corresponding to an eigenvalue \(l_i\)
in decreasing order. \$\mathbf{z} \in \mathbb{R}\^{}p \$ and \(p\) is
the number of different series being measured.

By partitioning the decomposition into the principal \(m\) and minor
\(p-m\) components, we obtain a factor analysis and error accordingly:
\[\mathbf{x}_t = (A_m|A^*_{p-m} )^T  \left( \frac{\mathbf{z}_{m,t}} {\mathbf{z}^*_{p-m,t}} \right) \]
\[ = A_m^T \mathbf{z}_{m,t} + (A^*_{p-m})^T \mathbf{z}^*_{p-m,t} \]
\[= \Lambda \mathbf{f}_t + \boldsymbol{\epsilon}_t \]

Substituting in the sample covariance gives us our estimates; for a
model with \(\Psi = \sigma^2 I\) this maximises the log-likelihood
function (PRML p.548).

\hypertarget{factor-analysis-and-probabilistic-graphical-models}{%
\subsubsection{Factor Analysis and Probabilistic Graphical
Models}\label{factor-analysis-and-probabilistic-graphical-models}}

\hypertarget{references}{%
\subsubsection{References}\label{references}}

Pattern Recognition and Machine Learning, Christopher Bishop


\end{document}
